
% !TEX encoding = UTF-8 Unicode 
% !TEX root = on_gates.tex

\clearpage
\section{The canonical gate}\index{canonical gate}
The canonical gate is a 3-parameter quantum logic gate that acts on two qubits~\cite{???,???,???}.
\[
\Gate{Can}&(t_x, t_y, t_z) 
\notag \\ = 
&\exp\Bigl(-i\frac{\pi}{2}  (t_x X\otimes X + t_y Y\otimes Y + t_z Z \otimes Z) \Bigr)
\]
Recall that $X=(\begin{smallmatrix}0 & 1 \\ 1 & 0\end{smallmatrix})$,
$Y=(\begin{smallmatrix}0 & \text{-}i \\ i & 0\end{smallmatrix})$, 
and $Z=(\begin{smallmatrix}1 & 0 \\ 0 & \text{-}1\end{smallmatrix})$ are the 1-qubit Pauli matrices, 
and that
\[
X\otimes X & = \begin{bsmallmatrix}0&0&0&+1\\0&0&+1&0\\0&+1&0&0\\+1&0&0&0 \end{bsmallmatrix}\ ,
\notag  \\
Y\otimes Y & = \begin{bsmallmatrix}0&0&0&-1\\0&0&+1&0\\0&+1&0&0\\-1&0&0&0 \end{bsmallmatrix} \ , \text{and}
\notag  \\
Z\otimes Z & = \begin{bsmallmatrix}+1&0&0&0\\0&-1&0&0\\0&0&-1&0\\0&0&0&+1 \end{bsmallmatrix} \ .
\notag
\]

Other parameterizations of the canonical gate are common in the literature. Often the sign is flipped, or the $\frac{\pi}{2}$ factor is absorbed into the parameters, or both. The parameterization used here the nice feature that it corresponds to powers of direct products of Pauli operators (up to phase) (see~\eqref{ZZ},\eqref{XX},\eqref{YY}).
$$
\adjustbox{scale=0.9}{\begin{quantikz}[thin lines, column sep=0.75em, row sep={2.5em,between origins}]
 & \gate[2]{\Gate{Can}(t_x, t_y, t_z)} & \qw \\
 &                              & \qw
\end{quantikz}}
\simeq
\adjustbox{scale=0.9}{\begin{quantikz}[thin lines, column sep=0.75em, row sep={2.5em,between origins}]
& \gate[2]{XX^{t_x}} &\gate[2]{YY^{t_y}} &\gate[2]{ZZ^{t_z}} & \qw \\
 &                          &  &  & \qw
\end{quantikz}}
$$
Here we use '$\simeq$' to indicate that two gates have the same unitary operator up to a global (and generally irrelevant) phase factor. 
\index{$\simeq$, equal up to global phase}
\index{$\loceq$, locally equivalent gates}
\index{local equivalence}
\index{phase}
% TODO: Cite B paper for \sim notation for locally equivalent gates?


The canonical gate is, in a sense, the elementary 2-qubit gate, since any other 2-qubit gate can be decomposed into a canonical gate, and
local 1-qubit interactions~\cite{???,???, Zhang2003a,Zhang2004a,Blaauboer2008a,Watts2013a}.
$$
\adjustbox{scale=0.9}{\begin{quantikz}[thin lines, column sep=0.75em, row sep={2.5em,between origins}]
& \gate[2]{U_0} & \qw \\
&  & \qw
\end{quantikz}}
\simeq
\adjustbox{scale=0.9}{\begin{quantikz}[thin lines, column sep=0.75em, row sep={2.5em,between origins}]
& \gate{K_1} & \gate[2]{\Gate{Can}(t_x, t_y, t_z)} & \gate{K_3} & \qw \\
& \gate{K_2} &                             & \gate{K_4} & \qw
\end{quantikz}}
$$
We will discuss the numerical decomposition of 2-qubit gates to canonical gates in section~\secref{sec:canonicaldeke}. For know it is sufficient to now that the non-local properties of every 2-qubit gate can be characterized by the 3-parameters of the corresponding canonical gate.
We'll use '$\loceq'$ to indicate that two gates are locally equivalent, in that they can be mapped to one another by local 1-qubit rotations.

The parameters of the canonical gate are periodic with period 4, or period 2 if we neglect a $-1$ global phase factor. Thus we can constrain each parameter to the range $[-1,1)$. Since $X\otimes X$,  $Y\otimes Y$, and $Z \otimes Z$ all commute, the parameter space has the topology of a 3-torus.
%
%TODO: Explain why they commute


\begin{center}
\begin{tikzpicture}[tdplot_main_coords, scale=2.5]
\draw (0,0,0) -- (2,0,0) -- (1,1,0)  -- cycle
      (0,0,0) -- (1,1,1) -- (1,1,0)  -- cycle
      (2,0,0) -- (1,1,1) -- (1,1,0)  -- cycle
      (2,0,0) -- (1,1,1) -- (1,1,0)  -- cycle;
%\draw (1,0,0) -- (1,1,0) -- (1,1,1) -- cycle;
%\draw (1,0,0) -- (1,1,0) -- (1,1,1) -- cycle;
\draw [dashed] (0,0,0) -- (2,0,0) -- (2,2,0)  -- (0,2,0) -- cycle
      (0,0,0) -- (0,0,2) -- (2,0,2)  -- (2,0,0) -- cycle
      (0,0,0) -- (0,0,2) -- (0,2,2)  -- (0,2,0) -- cycle;
\draw [dashed] (2,2,2) -- (2,2,0);
\draw [dashed] (2,2,2) -- (2,0,2);
\draw [dashed] (2,2,2) -- (0,2,2);

\draw [dotted] (0,0,0) -- (2,2,2);
\draw [dotted] (0,0,2) -- (2,2,0);
\draw [dotted] (0,2,0) -- (2,0,2);
\draw [dotted] (2,0,0) -- (0,2,2);

\end{tikzpicture}
\end{center}

By applying local gates we can decrement any one of the canonical gate's parameters,
$$
\adjustbox{scale=0.8}{\begin{quantikz}[thin lines, column sep=0.75em,row sep={2.5em,between origins}]
& \gate{Y} & \gate[2]{\text{Can}(t_{x},t_{y},t_{z})} & \gate{Z} & \qw \\
& \gate{Y} &  & \gate{Z} & \qw
\end{quantikz}
}
=
\adjustbox{scale=0.8}{\begin{quantikz}[thin lines, column sep=0.75em,row sep={2.5em,between origins}]
& \gate[2]{\text{Can}(t_{x} - 1,t_{y},t_{z})} & \qw \\
&  & \qw
\end{quantikz}
} \ ,
$$
we can flip the signs on any pair of parameters,
$$
\adjustbox{scale=0.8}{\begin{quantikz}[thin lines, column sep=0.75em,row sep={2.5em,between origins}]
& \gate{Z} & \gate[2]{\text{Can}(t_{x},t_{y},t_{z})} & \gate{Z} & \qw \\
& \qw &  & \qw & \qw
\end{quantikz}
}
=
\adjustbox{scale=0.8}{\begin{quantikz}[thin lines, column sep=0.75em,row sep={2.5em,between origins}]
& \gate[2]{\text{Can}(- t_{x},- t_{y},t_{z})} & \qw \\
&  & \qw
\end{quantikz}
} \ ,
$$
or we can swap any pair of parameters,
$$
\adjustbox{scale=0.8}{\begin{quantikz}[thin lines, column sep=0.75em,row sep={2.5em,between origins}]
& \gate{S} & \gate[2]{\text{Can}(t_{x},t_{y},t_{z})} & \gate{S^\dagger} & \qw \\
& \gate{S} &  & \gate{S^\dagger} & \qw
\end{quantikz}
}
= 
\adjustbox{scale=0.8}{\begin{quantikz}[thin lines, column sep=0.75em,row sep={2.5em,between origins}]
& \gate[2]{\text{Can}(t_{y},t_{x},t_{z})} & \qw \\
&  & \qw
\end{quantikz}
} \ .
$$

Because of these relations the canonical coordinates of any given 2-qubit gate are not unique since we have considerable freedom in the prepended and appended local gates. To remove these symmetries we can constraint the canonical parameters to a ``Weyl chamber''~\cite{???,???}. \index{Weyl chamber}
\begin{equation}
(\tfrac{1}{2} \ge  t_x \ge t_y \ge t_z \ge 0) \cup (\tfrac{1}{2} \ge (1-t_x) \ge t_y \ge t_z > 0 )
\label{WeylChamber}
\end{equation}
This Weyl chamber forms a  trirectangular tetrahedron.  All gates in the Weyl chamber are locally inequivalent (They cannot be obtained from each other via local 1-qubit gates). The net of the Weyl chamber is illustrated in Fig.~\ref{weyl_fig}, and the coordinates of many common 2-qubit gates are listed in table~\ref{weyl_table}. 

%With these symmetries we can constrain the canonical gate to the Weyl chamber.

There is an additional symmetry across the bottom face of the chamber. Gates located at $\Gate{Can}(t_x, t_y, 0)$ are locally equivalent to $\Gate{Can}(1-t_x, t_y, 0)$, since we can now flip the sign of $t_x$ without changing the other parameters.
$$
\adjustbox{scale=0.8}{\begin{quantikz}[thin lines, column sep=0.75em,row sep={2.5em,between origins}]
& \qw & \gate[2]{\text{Can}(t_{x},t_{y},0)} & \gate{Z} & \gate{Y} & \qw \\
& \gate{Y} &  & \qw & \gate{Z} & \qw
\end{quantikz}
}
= 
\adjustbox{scale=0.8}{\begin{quantikz}[thin lines, column sep=0.75em,row sep={2.5em,between origins}]
& \gate[2]{\text{Can}(1 - t_{x},t_{y},0)} & \qw \\
&  & \qw
\end{quantikz}
} 
$$

%Code for performing a canonical-decomposition, and therefore of determining the Weyl coordinates, can be found in the decompositions subpackage of {\tt QuantumFlow}~\cite{QuantumFlow}.
%
 Neighboring canonical gates can be merged by summing the parameters. 
$$
\adjustbox{scale=0.8}{\begin{quantikz}[thin lines, column sep=0.75em,row sep={2.5em,between origins}]
& \gate[2]{\text{Can}(s_{x},s_{y},s_{z})} & \gate[2]{\text{Can}(t_{x},t_{y},t_{z})} & \qw \\
&  &  & \qw
\end{quantikz}
}
= 
\adjustbox{scale=0.8}{\begin{quantikz}[thin lines, column sep=0.75em,row sep={2.5em,between origins}]
& \gate[2]{\text{Can}(s_{x} + t_{x},s_{y} + t_{y},s_{z} + t_{z})} & \qw \\
&  & \qw
\end{quantikz}
}
$$
Taking the Hermitian conjugate of the canonical gate simple inverts the parameters $\Gate{Can}(t_x, t_y, t_z)^\dagger  = \Gate{Can}(-t_x, -t_y, -t_z)$, and more generally powers of the canonical gate multiply the parameters, $\Gate{Can}(t_x, t_y, t_z)^c = \Gate{Can}(c t_x, c t_y, c t_z)$.


\begin{figure*}[tp]
\begin{center}
\begin{tikzpicture}[tdplot_main_coords, scale=6.0]
\draw (0,0,0) -- (2,0,0) -- (1,1,0)  -- cycle
      (0,0,0) -- (1,1,1) -- (1,1,0)  -- cycle
      (2,0,0) -- (1,1,1) -- (1,1,0)  -- cycle
      (2,0,0) -- (1,1,1) -- (1,1,0)  -- cycle;
\draw (1,0,0) -- (1,1,0) -- (1,1,1) -- cycle;
%\draw [ultra thick, Maroon] (0, 0, 0) -- (1,1,1) -- (2, 0, 0);

\node (SWAP) at (1, 1, 1) {};
\node (SWAP_L) at (1+0.25, 1, 1) {${\Gate{SWAP}}$};
\draw[thin, ->] (SWAP_L) -- (SWAP);

\node (SRSWAP) at (0.5, 0.5, 0.5) {};
\node (SRSWAP_L) at (0.5-0.25, 0.5, 0.5) {${\Gate{\sqrt{SWAP}}}$};
\draw[ultra thin, ->] (SRSWAP_L) -- (SRSWAP);

\node (SRSWAPI) at (1.5, 0.5, 0.5) {};
\node (SRSWAPI_L) at (1.5+0.25, 0.5, 0.5) {${\Gate{\sqrt{SWAP}^\dagger}}$};
\draw[ultra thin, ->] (SRSWAPI_L) -- (SRSWAPI);

\node (I2) at (2, 0, 0) {};
\node (I2_L) at (2, 0, -0.25) {${{I}}$};
\draw[ultra thin, ->] (I2_L) -- (I2);

\node (I) at (0, 0, 0) {};
\node (I_L) at (0, 0, -0.25) {${{I}}$};
\draw[ultra thin, ->] (I_L) -- (I);

\node (CNOT) at (1, 0, 0) {};
\node (CNOT_L) at (1, 0, -0.25) {${\Gate{CNOT}}$};
\draw[ultra thin, ->] (CNOT_L) -- (CNOT);

\node (SRCNOT) at (0.5, 0, 0) {};
\node (SRCNOT_L) at (0.5, 0, -0.25) {${\Gate{CV}}$};
\draw[ultra thin, ->] (SRCNOT_L) -- (SRCNOT);

\node (SRCNOT2) at (1.5, 0, 0) {};
\node (SRCNOT2_L) at (1.5, 0, -0.25) {${\Gate{CV}}$};
\draw[ultra thin, ->] (SRCNOT2_L) -- (SRCNOT2);

\node (iSWAP) at (1, 1, 0) {};
\node (iSWAP_L) at (1, 1, -0.25) {${\Gate{iSWAP}}$};
\draw[ultra thin, ->] (iSWAP_L) -- (iSWAP);

\node (B) at (1, 0.5, 0) {};
\node (B_L) at (1, 0.5, -0.25) {${\Gate{B}}$};
\draw[ultra thin, ->] (B_L) -- (B);

\node (ECP) at (1, 0.5, 0.5) {};
\node (ECP_L) at (1-0.25, 0.5, 0.5) {${\Gate{ECP}}$};
\draw[ultra thin, ->] (ECP_L) -- (ECP);

\node (QFT) at (1, 1, 0.5) {};
\node (QFT_L) at (1+0.25, 1, 0.5) {${\Gate{QFT}}$};
\draw[ultra thin, ->] (QFT_L) -- (QFT);

\node (SRiSWAP) at (0.5, 0.5, 0) {};
\node (SRiSWAP_L) at (0.5, 0.5, -0.25) {${\Gate{\sqrt{iSWAP}}}$};
\draw[ultra thin, ->] (SRiSWAP_L) -- (SRiSWAP);

\node (SRiSWAP2) at (1.5, 0.5, 0) {};
\node (SRiSWAP2_L) at (1.5, 0.5, -0.25) {${\Gate{\sqrt{iSWAP}}}$};
\draw[ultra thin, ->] (SRiSWAP2_L) -- (SRiSWAP2);

\draw[thick,->] (0,0,0) -- (0.25,0,0) node[anchor=north east]{${t_x}$};
\draw[thick,->] (0,0,0) -- (0,0.25,0) node[anchor= west]{${t_y}$};
\draw[thick,->] (0,0,0) -- (0,0,0.25) node[anchor=south]{${t_z}$};

\end{tikzpicture}
\end{center}

\caption{Location of the 11 principal 2-qubit gates in the Weyl chamber. All of these gates have coordinates of the form $\Gate{CAN}(\sfrac{1}{4}k_x, \sfrac{1}{4}k_y, \sfrac{1}{4}k_z)$, for integer $k_x$, $k_y$, and $k_z$.
Note there is a symmetry on the bottom face such that $CAN(t_x, t_y, 0) \loceq CAN(\half-t_x, t_y, 0)$.
}
\end{figure*}

\def\arraystretch{1.5}
\begin{table}[tp]
\caption{Canonical coordinates of common 2-qubit gates}
\label{weyl_table}
\begin{threeparttable}
%\begin{center}
\centering
\begin{tabular}{lccccccc}
		\text{Gate}		& $t_x$ 	& $t_y$	& $t_z$ & & $t'_x$ 	& $t'_y$	& $t'_z$	\\
				& $\leq$\half & & &  &$>$\half & & \\ 
				& $\qquad$& & $\qquad$& $\qquad$& $\qquad$&  $\qquad$& $\qquad$\\
% Points \\
$\Gate{I_2}$						& 0		& 0		& 0	& & 1 &0&0	\\
$\Gate{CNot}$  / $\Gate{CZ}$ / \Gate{MS}	&\half	& 0		& 0		\\
$\Gate{iSwap}$ / $\Gate{DCNot}$ &\half	& \half		& 0		& & $\tfrac{3}{4}$ & \half & 0	\\
$\Gate{Swap}$  					&\half	& \half		& \half		\\
\\
CV					&$\tfrac{1}{4}$	& $0$		& 0		& & $\tfrac{3}{4}$ & 0 & 0	\\
$\sqrt{\Gate{iSwap}}$  			&$\tfrac{1}{4}$	& $\tfrac{1}{4}$		& 0		& & $\tfrac{3}{4}$ & $\tfrac{1}{4}$ & 0	\\
${\Gate{DB}}$  					&$\tfrac{3}{8}$	& $\tfrac{3}{8}$		& 0		& & $\tfrac{5}{8}$ & $\tfrac{3}{8}$ & 0	\\
$\sqrt{\Gate{Swap}}$  			&$\tfrac{1}{4}$	& $\tfrac{1}{4}$		& $\tfrac{1}{4}$		\\
$\sqrt{\Gate{Swap}}^\dagger$  	& & & & &$\tfrac{3}{4}$	& $\tfrac{1}{4}$		&$\tfrac{1}{4}$	\\
\\
$B$  							&\half	& $\tfrac{1}{4}$		& 0		\\
$\Gate{ECP}$  					&\half	& $\tfrac{1}{4}$		&  $\tfrac{1}{4}$	\\
$\Gate{QFT_2}$  				&\half	& \half		&  $\tfrac{1}{4}$	\\
$\Gate{Sycamore}$				&\half	& \half		&  $\tfrac{1}{12}$	\\
\\
% Edges
Ising / $\Gate{CPhase}$	& $t$ & 0 & 0 \\
$\Gate{XY}$	& $t$ & $t$ & 0 & & $t$ & 1-$t$ & 0  \\
Exchange	/ $\Gate{Swap}^\alpha$	& $t$ & $t$ & $t$ & & $t$ & 1-$t$ & 1-$t$ \\
$\Gate{PSwap}$ 	& \half & \half & $t$ \\
\\
% Surfaces
Special orthogonal 	& $t_x$ & $t_y$ & 0 \\
Improper orthogonal 	& \half & $t_y$ & $t_z$ \\
\Gate{XXY} 	&$t$ & $t$ & $\delta$ & &$t$ & 1-$t$ & $\delta$ \\
			& $\delta$ &  $t$ & $t$ & & $\delta$ &  $t$ & $t$  \\					

\end{tabular}
%\end{center}
%\begin{tablenotes}
%\item[]\small Note:  There is a symmetry on the base of the Weyl chamber such that $(t_x, t_y, 0)$ is equivalent to $(1-t_x, t_y, 0)$. I've  listed both coordinates for clarity, although the formal definition of the Weyl chamber~\eqref{WeylChamber} excludes the second form.
%\end{tablenotes}
\index{Weyl chamber}
\index{canonical coordinates}
\index{Swap gate}
\index{quantum Fourier transform}
\index{ECP gate}
\index{square-root Swap gate}
\index{inverse square-root Swap gate}
\index{square-root iSwap gate}
\index{iSwap gate}
\index{identity gate! 2-qubits}
\index{B gate}
\index{CNot gate}
\index{CZ gate}
\index{DCNot gate}
\index{CV gate}
\index{Dagwood Bumstead gate}
\end{threeparttable}
\label{default}
\end{table}%
