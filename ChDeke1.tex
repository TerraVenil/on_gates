
% !TEX encoding = UTF-8 Unicode 
% !TEX root = on_gates.tex


\clearpage
\section{Decomposition of 1-qubit gates}
A general 1-qubit gate corresponds to some 2 by 2 unitary matrix,
\[
V = 
e^{i\alpha}\begin{bsmallmatrix}
a & -b^* \\
b & a^*
\end{bsmallmatrix}\]
where $\alpha$ is real, $a$ and $b$ are complex with $|a|^2 + |b|^2 = 1$.
Given such a generic unitary, we would like to represent the gate using standard parameterized gates.   

The first step to deke\endnote{
{\sl deke} {\sl |dek|} verb --- To decompile, deconstruct, or decompose.
\\ 1995  Neal Stephenson {\sl The Diamond Age} ``We gotta deke all this stuff now'' Easy come, easy go.
} a gate is to extract the phase factor $\alpha$,
\[
V = e^{i\alpha} U
\]
so that $V$ is a special unitary matrix with $\det V=1$. In general, if we multiply a special unitary matrix by a complex phase $c$ then $\det cV=c^k$ where $k$ is the rank of the matrix, i.e. $k=2^n$ for n qubit. (The determinate is the product of the eigenvalues, and multiplying a matrix by a constant multiplies each of the eigenvalues). [TODOO: Wordsmith]

Thus the determinate of U is $\det U=e^{i 2 \alpha }$, and we can extract the phase factor $\alpha$ with some trigonometry.
\[
\alpha = \half \text{arctan2}(\Im(\det U), \Re(\det U))
\]

\todo{Explain arctan2}


\subsection{Bloch rotation decomposition}

\[
R_{\vec{n}}(\theta) =
\begin{bmatrix*}
	\cos(\half\theta) - i n_z \sin(\half\theta)  &
	- n_y \sin(\half\theta)-i n_x \sin(\half\theta)  \\
	n_y \sin(\half\theta)-i n_x \sin(\half\theta)   & 
	\cos(\half\theta) + i n_z \sin(\half\theta)
\end{bmatrix*}
\]

\[
	N &= \sqrt{(\Im V_{0,1})^2 + (\Re V_{0,1})^2 + (\Im V_{0,0})^2} \\
	n_x & = -\Im V_{0,1}/N  \notag \\
	n_y & = - \Re V_{0,1}/N \notag \\
	n_z & = - \Im V_{0,0}/N \notag \\
s = \sin(\half\theta) &= - \Im V_{0, 0} / n_z \notag \\ 
c = \cos(\half\theta)  &= \Re V_{0, 0}  \notag \\
\theta &= 2\ \text{arctan2}(s, c) \notag
\]


%-\Im V_{0,1} - \Re V_{0,1} - \Im U_{0,0}
%
%    U = asarray(gate.asoperator())
%    U /= np.linalg.det(U) ** (1/2)
%
%    nx = - U[0, 1].imag
%    ny = - U[0, 1].real
%    nz = - U[0, 0].imag
%    N = np.sqrt(nx**2 + ny**2 + nz**2)
%    if N == 0:  # Identity
%        nx, ny, nz = 1, 1, 1
%    else:
%        nx /= N
%        ny /= N
%        nz /= N
%    sin_halftheta = N
%    cos_halftheta = U[0, 0].real
%    theta = 2 * np.arctan2(sin_halftheta, cos_halftheta)


\subsection{Z-Y decompositions}
\todo{Naming. Pauli deke? Euler? Z-Y-Z or Z-Y?}
\endnote{The Z-Y decomposition is apparently of relatively ancient origin, long know in the theory of light polarization~\cite{Sousa2006a}}

\[
U =  e^{i\alpha}\ R_z(\theta_2)\ R_y(\theta_1)\ R_z(\theta_0)
\]
Or in cirucit notation.
\begin{center}
\adjustbox{scale=0.75}{\begin{quantikz}[thin lines, column sep=0.75em,row sep={2.5em,between origins}]
& \gate{U} & \qw 
\end{quantikz}}
$=$
\adjustbox{scale=0.8}{\begin{quantikz}[thin lines, column sep=0.75em,row sep={2.5em,between origins}]
& \gate{R_z(\theta_{0})} & \gate{R_y(\theta_{1})} & \gate{R_z(\theta_{2})} & \gate{\text{Ph}(\alpha)} & \qw
\end{quantikz}
}
\end{center}
Note that we have numbered the three angles in chronological order, and recall that time runs right-to-left in operator notation, but left-to-right in circuit notation. 

\[
V = 
\begin{bmatrix}
+e^{+i(+\half\theta_2 + \half\theta_0)} \cos(\half\theta_1) 
& -e^{-i(-\half\theta_2 + \half\theta_0)} \sin(\half\theta_1) 
\\
+e^{+i(+\half\theta_2 - \half\theta_0)} \sin(\half\theta_1) 
& +e^{-i(+\half\theta_2 +\half\theta_0)} \cos(\half\theta_1)
\end{bmatrix}
\]


The value of $\theta_1$ can be calculated from the absolute value of either the diagonal or off-diagonal elements, provided those entries aren't close to zero.
For instance, the Z-gate has zero off-diagonal entries, whereas the X-gate has zeros on the diagonal. But the diagonal and off-diagonal entries can't approach zero at the same time.  So to calculate $\theta_1$ with greatest numerical accuracy, we use whichever element has the largest absolute value.
\[
\theta_1 = 
  \begin{cases}
		2 \arccos(|U_{00}|), & |U_{00}| \geq |U_{01}|\\	
		2 \arcsin(|U_{01}|), & |U_{00}| < |U_{01}|		
	\end{cases} 
\]

Having extracted $\theta_1$, we can now calculate the sum $\theta_0+\theta_1$ from $U_{11}$ using the $\text{arctan2}$ function, \eqref{arctan2},
\[
 \theta_0 + \theta_2 =  
  2\ \text{arctan2}\Left(\Im(\frac{U_{11}}{\cos(\half\theta_1)}),\ \Re(\frac{U_{11}}{\cos(\half\theta_1)})\Right)
\ .
\]
except if $\cos(\half\theta_1)=0$ then $\theta_0 + \theta_2 = 0$.

Similarly we can extract the difference  $\theta_0 - \theta_2$ from $U_{10}$.
\[
 \theta_0 - \theta_2 = 2\ \text{arctan2}\Left(\Im(\frac{U_{10}}{\sin(\half\theta_1)}),\ \Re(\frac{U_{10}}{\sin(\half\theta_1)})\Right)
 \, 
\]
again with an exception that if $\sin(\half\theta_1)=0$ then $\theta_0 - \theta_2 = 0$.
%
Taking the sum and differences of \eqref{???} and \eqref{???} yields $\theta_0$ and $\theta_2$, which completes the decomposition. 

Instead of rotation gates, we could express the same decomposition as Pauli-power gates with a simple reparameterization. 
\[
U = & e^{i\alpha'}\ Z^{t_2}\ Y^{t_1}\ Z^{t_0}
%= e^{i\alpha}\ R_z(\theta_2)\ R_y(\theta_1)\ R_z(\theta_0)
\\
& \alpha' = \alpha + ??? \notag \\
& t_0 = \theta_0/\pi \notag\\
& t_1 = \theta_1/\pi \notag\\
& t_2 = \theta_2/\pi \notag
\]

For the IBM-??? architecture the natural 1-qubit gates are $R_z$ and $V$ (the square root of $X$)~\cite{???}. There isn't a direct access to $R_y$ rotations or general $R_x$ rotations, but this is only a minor restriction since $R_z(\theta) = V^\dagger R_y(\theta) V$, (Ref) and we can therefore decompose 1-qubit gates to a 5-gate sequence.
\[
U =  e^{i\alpha}\ R_z(\theta_2)\ V^\dagger\ R_z(\theta_1)\ V\  R_z(\theta_0)
\]
\todo{Circuit}
\todo{Name of IBM architecture? Check location of dagger on Vs}
% ZVZVZ deke



% ================================================================================
\subsection{ABC decomposition}

Another useful trick is to decompose a 1-qubit unitary as $U = e^{i\alpha} AXBXC $~\cite{Barenco1995b}, where
\[
A &= R_z(\theta_2) \ R_y(\half\theta_1) \ ,\notag \\
B &=  R_y(-\half\theta_1)\ R_z(-\half\theta_0 -\half\theta_2)\ ,\notag \\
C &= R_z(\half\theta_0 -\half\theta_2) \ . \notag
\]
Note that $ABC=I$. We will use this relation in the decomposition of 2 qubit gates \todo{Forward ref}. 


This decomposition is the really just the Z-Y decomposition in slight disguise.
\[
U &= e^{i\alpha} AXBXC 
\notag
\\ 
& =
	e^{i\alpha} 
	R_z(\theta_2) \ R_y(\half\theta_1)
	\ X \ R_y(-\half\theta_1)\ X 
	\ X\ R_z(-\half\theta_0 -\half\theta_2)\ X\
	R_z(\half\theta_0 -\half\theta_2)
\notag \\
& =
	e^{i\alpha} 
	R_z(\theta_2) \ R_y(\half\theta_1)
	\ \phantom{X} \ R_y(+\half\theta_1)\ \phantom{X}
	\ \phantom{X}\ R_z(+\half\theta_0 +\half\theta_2)\ \phantom{X}\
	R_z(\half\theta_0 -\half\theta_2)
\notag \\
&=
	e^{i\alpha}\
	R_z(\theta_2) \ R_y(\theta_1)
	R_y(\theta_1)\  
	R_z(\theta_0)
\]
In the second line we insert the identity $I=XX$, and then recognize that $X\ R_z(\theta)\ X = R_x(-\theta)$ and $X\ R_y(\theta)\ X = R_y(-\theta)$. We can understand these relations by looking at the Bloch sphere. The $X$ gate isa half turn rotation about the $x$ axis, so the $z$ and $y$ axes are inverted, and the respective rotation gates induce an anti-clockwise rather than clockwise rotations relative to the original axes. 

 


% ================================================================================
\subsection{General Pauli-rotation decompositions}
% TODO: Check terminology, Euler decomposition

\[
U &=  R_x(\theta_2) R_y(\theta_1) R_x(\theta_0)
\]

\[
V = C U C^\dagger &= C R_y(\theta_2)  C^\dagger\ C R_z(\theta_1)C^\dagger\ C R_y(\theta_0) C^\dagger
\\
& = R_z(\theta_2) R_y(\theta_1) R_z(\theta_0)
\]

We want the single qubit gate $C$ that moves the +Y axis to +X, but leaves the Z axis alone. This is a Clifford gate, and consulting table~\ref{tab:Clifford1q} we see that the gate is $S^\dagger$. 

\begin{table}[htp]
\caption{Euler decompositions}
\label{tab:geneuler}
\begin{center}
\begin{tabular}{cl}
Euler decomposition & Similarity transform to Z-Y-Z \\
    X-Y-X & $h^\dagger$ \\
    X-Z-X & $R_n(+\tfrac{2\pi}{3}, \tfrac{1}{\sqrt{3}}, \tfrac{1}{\sqrt{3}}, \tfrac{1}{\sqrt{3}})$ \\
    Y-X-Y & $R_n(-\tfrac{2\pi}{3}, \tfrac{1}{\sqrt{3}}, \tfrac{1}{\sqrt{3}}, \tfrac{1}{\sqrt{3}})$ \\
    Y-Z-Y & $R_n(\pi, 0, \tfrac{1}{\sqrt{2}}, \tfrac{1}{\sqrt{2}}$ \\
    Z-X-Z & $S^\dagger$ \\
    Z-Y-Z & $I$
\end{tabular}
\end{center}
\end{table}


% ================================================================================

\subsection{Kronecker decomposition}
\cite{VanLoan1993a, VanLoan2000a}
% Cite: Approximation with Kronecker Products
% Pitsianis-Van Loan algorithm 

\[
C = A\otimes B
\]

\[
C_{mp;nq} = A_{mn}\otimes B_{pq} = [A_{mn}\cdot B_{pq}]^{T_{n\leftrightarrow p}} 
\]

% TODO: Check code.
\begin{verbatim}
import numpy as np
def nearest_kronecker_product(C):
    C = C.reshape(2, 2, 2, 2)
    C = C.transpose(0, 2, 1, 3)  
    C = C.reshape(4, 4)

    u, sv, vh = np.linalg.svd(C)
    
    A = np.sqrt(sv[0]) * u[:, 0].reshape(2, 2)
    B = np.sqrt(sv[0]) * vh[0, :].reshape(2, 2)

    return A, B
\end{verbatim}
We first shape C to a 4th order tensor, so that in the next line we can undo the axes transposition, before reshaping to a matrix. The singular value decomposition takes this matrix apart, and we retain the rank-1 approximation, retaining only the largest singular value and corresponding left and right singular vectors. We reshape the singular vectors to matrices to obtain our result. (multiplied by the   
$A\otimes B$ is the closest Kronicker product to C in the Frobenius norm.

