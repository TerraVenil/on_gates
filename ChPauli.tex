
% !TEX encoding = UTF-8 Unicode 
% !TEX root = on_gates.tex

\clearpage

\section{Pauli Group and Pauli Algebra}

Recall the 4 1-qubit Pauli operators: $I=[\begin{smallmatrix} 0& 0 \\ 0 & 1 \end{smallmatrix}]$,
$X=[\begin{smallmatrix}0 & 1 \\ 1 & 0 \end{smallmatrix}]$, $Y=[\begin{smallmatrix}0 & -i \\ i & 0 \end{smallmatrix}]$, $Z=[\begin{smallmatrix}1 & 0 \\ 0 & -1 \end{smallmatrix}]$.
\[
X^2 = Y^2 = Z^2 = I  \\
XY = -YX = iZ \notag\\
ZX = -XZ = iY \notag\\
YZ = -ZY = iX \notag
\]
Every pair of Pauli matrices either commutes or anti-commutes. 

The {\sl Pauli group} of 1 qubit operators  consists of the 4 Pauli operators  multiplied by factors of $\pm 1$ or $\pm i$. This extra phase ensures that these 16 elements form a group under matrix multiplication. The Pauli group $P_n$ of $n$ qubit operators contains $4^{n+1}$ elements is formed from the 4 phase factors and tensor products of 1-qubit Pauli matrices, 
\[
P_n = \{\pm 1, \pm i\} \times \{I, X, Y, Z\}^{\otimes n}
\]


